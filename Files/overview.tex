\subsection {Perspective}

In this paragraph we shall deepen the shared phenomena mentioned in the previous section, providing an oversight of the domain model on different levels of specification, by means of class and state diagrams.

\begin{itemize}

\item \textbf{Login (controlled by the world and observed by the machine)}: On every opening of the app, users must insert their phone number, then the code generated by the server and sent to them by SMS. Authorities enter using their ID code and password provided to them in a previous moment.

\item \textbf{File report (controlled by the world and observed by the machine)}: Users must take a photo of the violation, position is obtained by the tracker in the phone, date and time are provided by the server, and a brief description can be attached. Nowithstanding that the system exploits computer vision and artificial intelligence to identify the license plate of the vehicle in the photos, the user can add the number to improve the accuracy of the report.In case of incomplete identification the user is warned and invited to retake the photo (Reports with no photo or incomplete plate identification are not accepted, and the user must start over). If the report complies with all this rules, it is sent to Safestreets' DBMS.

\item \textbf{Inspect report (controlled by the world and observed by the machine)}: Authorities have access to all the pending reports of their city. They can discard the report (its data is erased by the DBMS) or inspect it. Exploiting Google Maps' API, they are guided to the position of the report. The result can be either positive (the violation is verified and a fine or a warning is issued) or negative. In the former option, the authority completes the report choosing between a list of possible violation and types of vehicle and closes the case, saving it on the database, in the latter one, the data is discarded.

\item \textbf{Show data of violations (controlled by the machine and observed by the world)}: All the successful reports are stored on the DB with statistics regarding every street. Streets are ordered by the number of violations committed during the previous month in decreasing orderd. Users can select a street and acknowledge the number of violation type by type, authorities see also the list of the plates of the offenders.

\item \textbf{Show data of accidents (controlled by the machine and observed by the world)}: If the \\ municipality offers data about accidents it is crossed with that of Safestreets, street by street. Streets are ordered by the number of accidents, those who have a number of accidents equal to or greater than the treshold (for example \textcolor{Red}{12}) are marked unsafe, and the most common type of violation is shown. The number of accidents can be seen by everyone.

\item \textbf{Show suggestions (controlled by the machine and observed by the world)}: Suggestion are formulated for unsafe streets based on the most common type of violation, they can be seen only by the authorities.

\end{itemize}

Following are the class and state diagrams:

%TODO