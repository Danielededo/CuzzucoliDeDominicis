\subsection{Purpose}

Safestreet is the name of an application aiming to offer principally the possibility to signal traffic or parking violation via the internet to public authorities.
This document aims to outline the functioning of said app, providing a deep insight of the software and supporting the stakeholders.

Safestreet offers, as a basic functionality, the possibility to fill in a form detailing one of the possible violations (e.g. illegal parking) to every single user. The application gives its users the possibility to compile a form complete with a brief description of the wrongdoing and several additional information. Said information consists of the position of the user at the time of the report (street name) and some pictures, to be attached to the form itself in some way (for example using the smartphone on which the app is installed). The form can then be either approved and investigated or discarded by the public authorities working at the interested area (in both cases the user is warned of the end result of their report). At the same moment Safestreet stores data received by those warnings and computes them to highlight areas in which a violation is most likely to occur, or to show which type of vehicle tends to be associated with different wrongdoings.

In addition to what is written above, the data mined from the warnings can be crossed with that offered by the municipality. If a specific part of the city presents an alarming number of reports it may mean that the infrastructures provided to the population are inadequate or that part of the citizens (especially the weaker ones such as bikers or the elderly) is not preserved enough, thus some suggestions can be made to the authorities in order to avoid further inconveniences.

\subsection{Goals}

The goals set are expressed from the point of view of the S2B:

\begin{itemize}
\item [G1]{The application has to store all the information about violations sent to it, until a ticket is either dropped or accepted by an authority}
\item [G2]{The system must accepts reports issued from every part of the covered area }
\item [G3]{The system must allow authorities to access reports in every part of the covered area}
\item [G4]{The application allows users and authorities to mine information on the system}
\item [G5]{The application identifies unsafe areas crossing its informations with those offered by the municipality}
\item [G6]{The application suggests possible solution to problems perceived after the crossing of information}
\end{itemize}

\subsection{Scope}

Safestreet poses itself as an intermediate between users and authorities.\\
%Users must be equipped with smartphones provided with camera and GPS tracker.\\
Users file reports via the mobile app. 
Authorities (e.g. a police person) examine the report and can discard it or investigate it, leading to a positive result (fine or warning) or negative one (no violation revealed).\\
Saved reports are analyzed and data extracted by them can be consulted both by users and authorities (at different levels).
If the municipality offers data about accidents, they are crossed with those obtained by SafeStreets and suggestion based on them are made. Data is visible to everyone (at different levels).

\subsubsection{World Phenomena}

\begin{itemize}

\item \textbf{Broken rules}: Drivers commit violations

\item \textbf{Observation}: Citizens observe possible violations

\item \textbf{Inspection}: Authorities inspect possible violations 

\item \textbf{Unfortunate events}: Accidents happen

\end{itemize}

\subsubsection{Machine Phenomena}

\begin{itemize}

\item \textbf{DBMS}: All operations to retrieve or store data

\item \textbf{Classification}: Statistics are redacted and streets ordered accordingly

\end{itemize}

\subsubsection{Shared Phenomena}

\subsubsection*{Controlled by the world and observed by the machine}

\begin{itemize}

\item \textbf{Login}: Citizens undergo the SMS procedure and authorities login using ID and password

\item \textbf{File report}: Citizens file a report

\item \textbf{Inspect report}: Authorities discard or complete a report

\end{itemize}

\subsubsection*{Controlled by the machine and observed by the world}

\begin{itemize}

\item \textbf{Show data of violations}: Violations are listed and acessible by everyone

\item \textbf{Show data of accidents}: Accidents are listed and acessible by everyone

\item \textbf{Show suggestions}: Suggestion are made according to accidents and violations, for the authorities to see

\end{itemize}

\subsection{Definitions, Acronyms, Abbreviations}

\subsubsection{Definitions}

\begin{itemize}

\item Smartphone: Device that allows for calls together with internet connectivity and GPS tracking

\item User: Citizen who has installed Safestreet on their smartphone

\item Authority: Public official or ent that monitors the territory and mantains the order

\item Violation: Act of breaking rules of the road code

\item Accident: Violent collision involving vehicles (motorized or not) and/or animate or inanimate objects
 
\end{itemize}

\subsubsection{Acronyms}
\begin{tabular}{|l|l|}
\hline
Acronym & Meaning \\ \hline
DB & Data Base \\ \hline
DBMS & Data Base Management System \\ \hline
API & Application Program Interface \\ \hline
UI & User Interface \\ \hline
OS & Operating System \\ \hline
RASD & Requirement Analysis and Specification Document \\ \hline
GPS & Global Positioning System \\
\hline
\end{tabular}

\subsubsection{Abbreviation}

\begin{itemize}

\item [\textbf{G.th}]: n-th Goal

\item [\textbf{D.th}]: n-th Domain Assumption

\item [\textbf{R.th}]: n-th Functional Requirement

\end{itemize}

\subsection{Reference Documents}

\begin{itemize}

\item Project assignment specifications:\cite{assignment.pdf}

\item Alloy Documentation:\cite{http://alloy.lcs.mit.edu/alloy/documentation.html}

\end{itemize}

\subsection{Document Structure}

\begin{itemize}

\item \textbf{Overall description}: a general description of the service is provided, together with a furter inspection on shared phenomena. Relevant functions of the System are explained. Finally, through the specifications of constraints, dependencies and assumptions, it is unveiled how the System is integrated in the real world.

\item \textbf{Specific requirements}: aimed to both designers and testers, all the aspects of the previous section are put into perspective. Requirements
are fully explained and divided by types and possible interactions with the System are investigated through use cases and sequence diagrams.

\end{itemize}