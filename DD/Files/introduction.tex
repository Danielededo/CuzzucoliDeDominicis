\subsection{Purpose}

This document outlines the SafeStreets service, both basic and advanced funcionalities, introduced in the corresponding RASD

\subsection{Scope}

SafeStreets is a service that basically allows users (\a.k.a. normal citizens) to upload reports about violations. This reports are seen by authorities and investigated, if deemed appropriate, or dropped if not. Every user is also allowed to inspect data regarding violations w.r.t. the area interested by them, with limitations based on the type of user exploiting the service.
In addition to the basic function, Safestreets can be used to acknowledge statistics about accidents in a fashion similar to that of the violations.

\subsection{Definitions, Acronyms, Abbreviations}

\subsubsection{Definitions}

\begin{itemize}

\item \textbf{Client:}  Piece of software or hardware that can access services offered by a server in different forms.

\item \textbf{Server:} Piece of software or hardware that offers different services (that can constitute a part or the entirety of an application) to one or more clients.

%TODO : add more definitions if necessary

\end{itemize}

\subsubsection{Acronyms}

\begin{tabular}{|l|l|}
\hline
Acronym & Meaning \\ \hline
DB & Data Base \\ \hline
DBMS & Data Base Management System \\ \hline
DD & Design Document \\ \hline
API & Application Program Interface \\ \hline
UI & User Interface \\ \hline
UX & User Experience \\ \hline
OS & Operating System \\ \hline
RASD & Requirement Analysis and Specification Document \\ \hline
GPS & Global Positioning System \\ \hline
OTP & One Time Password \\ 
\hline

%TODO : add more acronyms if necessary

\end{tabular}

\subsubsection{Abbreviations}

\begin{itemize}

\item [\textbf{G.th}]: n-th Goal

\item [\textbf{D.th}]: n-th Domain Assumption

\item [\textbf{R.th}]: n-th Functional Requirement

\end{itemize}

%TODO : yet another one

\subsection{Revision history}

%No history Yet

\subsection{Reference Documents}

\begin{itemize}

\item Project assignment specifications:\cite{ASSIGNMENT}

\item UML: \cite{UML}

\item DD to be analyzed: \cite{DD}

\end{itemize}


\subsection{Document Structure}

\begin{itemize}

\item \textbf{Introduction:} summary of the concepts already expressed in the RASD document.
\item \textbf{Architectural Design:} detailed description of the architectural design w.r.t components and design patterns.
\item \textbf{User Interface Design:} addition details on the UI previously sketched in the RASD document by means of UX modeling.
\item \textbf{Requirements Traceability:} analysis on the requirements of the RASD and how they are satisfied by the design choices of the DD.
\item \textbf{Implentation, Integration and Test plan:} showing implementation and integration of subcomponents in the defined order and giving details on the subsequential testing for the integration.

\end{itemize}