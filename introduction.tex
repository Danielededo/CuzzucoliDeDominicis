\subsection{Purpose}

Safestreet is the name of an application aiming to offer principally the possibility to signal traffic or parking violation via the internet to public authorities.
This document aims to outline the functioning of said app, providing a deep insight of the software and supporting the stakeholders.

Safestreet offers, as a basic functionality, the possibility to fill in a form detailing one of the possible violations (e.g. illegal parking) to every single user. The application gives its users the possibility to compile a form complete with a brief description of the wrongdoing and several additional information. Said information consists of the position of the user at the time of the report (street name) and some pictures, to be attached to the form itself in some way (for example using the smartphone on which the app is installed). The form can then be either approved and investigated or discarded by the public authorities working at the interested area (in both cases the user is warned of the end result of their report). At the same moment Safestreet stores data received by those warnings and computes them to highlight areas in which a violation is most likely to occur, or to show which type of vehicle tends to be associated with different wrongdoings.

In addition to what is written above, the data mined from the warnings can be crossed with that offered by the municipality. If a specific part of the city presents an alarming number of reports it may mean that the infrastructures provided to the population are inadequate or that part of the citizens (especially the weaker ones such as bikers or the elderly) is not preserved enough, thus some suggestions can be made to the authorities in order to avoid further inconveniences.

\subsection{Goals}

The goals set are expressed from the point of view of the S2B:

\begin{itemize}


\item [G1]{The application has to store all the information about violations sent to it, until a ticket is either dropped or accepted by an authority}

\item [G2]{The system must accepts reports issued from every part of the covered area }

\item [G3]{The system must allow authorities to access reports in every part of the covered area}

\item [G4]{The application allows users and authorities to mine information on the system}

\item [G5]{The application identifies unsafe areas crossing its informations with those offered by the municipality}

\item [G6]{The application suggests possible solution to problems perceived after the crossing of information}

\end{itemize}

\subsection{Scope}

Safestreet poses itself as an intermediate between users and authorities.

Users must be equipped with smartphones provided with camera and GPS tracker.\\
On the opening of the app, users must insert their phone number, then the code generated by the server and sent to them by SMS.To file a report users must take a photo of the violation, position is obtained by the tracker in the phone, date and time are provided by the server, and a brief description can be attached. Nowithstanding that the system exploits computer vision and artificial intelligence to identify the license plate of the vehicle in the photos, the user can add the number to improve the accuracy of the report. In case of partial identification the user is warned and invited to retake the photo. Reports with no photo or partial identification are not accepted, and the user must start over.

Authorities (for example a policeperson) have access to all the pending reports of their city. They can discard the report (its data is not saved) or inspect it; the result can be either positive (the violation is verified and a fine or a warning is issued) or negative. In the former option, the authority completes the report choosing between a list of possible violation and types of vehicle and closes the case, saving it on the database, in the latter one, the data is discarded.\\

All the successful reports are stored on the database with statistics regarding every street. Streets are ordered by the number of violations committed during the previous month in decreasing orderd. Users can select a street and acknowledge the number of violation type by type, authorities see also the list of the plates of the repeated offenders (those who have broken rules two times or more on the same road)\\

If the municipality offers data about accidents it is crossed with that of Safestreets, street by street. Streets are ordered by the number of accidents, those who have a number of accidents equal to or greater than the treshold (for example \textcolor{Red}{12}) are marked unsafe, and the most common type of violation is shown. Suggestion are formulated based on the violation and are visible only by the authorities, while the number of accidents can be seen by everyone.